%% ****** Start of file aiptemplate.tex ****** %
%%
%%   This file is part of the files in the distribution of AIP substyles for
% %  REVTeX4.
%%   Version 4.1 of 9 October 2009.
%%
%
% This is a template for producing documents for use with
% the REVTEX 4.1 document class and the AIP substyles.
%
% Copy this file to another name and then work on that file.
% That way, you always have this original template file to use.

\documentclass[aip,graphicx]{revtex4-1}
%\documentclass[aip,reprint]{revtex4-1}
\usepackage[draft]{todonotes}

\draft % marks overfull lines with a black rule on the right

\begin{document}

% Use the \preprint command to place your local institutional report number 
% on the title page in preprint mode.
% Multiple \preprint commands are allowed.
%\preprint{}

\title{Modeling vertical-axis cross-flow turbine wakes}

% repeat the \author .. \affiliation  etc. as needed
% \email, \thanks, \homepage, \altaffiliation all apply to the current author.
% Explanatory text should go in the []'s, 
% actual e-mail address or url should go in the {}'s for \email and \homepage.
% Please use the appropriate macro for the type of information

% \affiliation command applies to all authors since the last \affiliation command. 
% The \affiliation command should follow the other information.

\author{Peter Bachant}
\email[]{pxl3@unh.edu}
%\homepage[]{Your web page}
%\thanks{}
%\altaffiliation{}
\affiliation{Center for Ocean Renewable Energy, University of New Hampshire, 
Durham, NH}

\author{Martin Wosnik} 
%\email[]{}
%\homepage[]{Your web page}
%\thanks{}
%\altaffiliation{}
\affiliation{Center for Ocean Renewable Energy, University of New Hampshire,
Durham, NH}

% Collaboration name, if desired (requires use of superscriptaddress option in
% \documentclass).
% \noaffiliation is required (may also be used with the \author command).
%\collaboration{}
%\noaffiliation

\date{\today}

\begin{abstract}
The wake of a vertical-axis cross-flow turbine (CFT) is modeled numerically via
several techniques.
\end{abstract}

\pacs{}% insert suggested PACS numbers in braces on next line

\maketitle %\maketitle must follow title, authors, abstract and \pacs
\listoftodos

\section{Introduction}

Araya et al. modeled the flow through a VAT array using leaky Rankine bodies
\cite{Araya2014}.

Questions addressed:

\begin{enumerate}

    \item Can 2-D RANS be used for array design?
    
    \item Can ALM--RANS be used for array design?

    \item Does 3-D RANS tell us more about what is happening, i.e, can it
    ``interpolate'' the experimental results?

\end{enumerate}

\section{Numerical models}

The flow field was modeled using the Reynolds-averaged Navier--Stokes equations,
employing two different turbulence models--Menter's $k$--$\omega$ SST
\cite{Menter1994} and the Spalart--Allmaras one equation model
\cite{Spalart1992}. The SST model was chosen due to its prominence in the
literature for simulating separating flows, which we assumed to be present in
the current problem in the form of dynamic stall. The Spalart--Allmaras model
was shown by Ferreira et al. \cite{Ferreira2007} to match experimental particle
image velocimetry (PIV) results for a CFT in dynamic stall, though this was a
somewhat low Reynolds number case ($5 \times 10^4$).

The RANS models employ a sliding mesh, where the turbine zone rotates at a mean
tip speed ratio $\lambda=1.9$ with an unsteadiness as the blade passage
frequency specified to mimic the experimental data.

To capture the boundary layer, 

\subsection{Software}

Simulations were run using the \texttt{pimpleDyMFoam} solver from the
open-source finite volume CFD package OpenFOAM, version 2.3.x.

\section{Model verification}

The $k$--$\omega$ SST and Spalart--Allmaras RANS model cases were verified for
convergence of the turbine mean power coefficient with respect to grid spacing
and timestep. The grid topology was fixed, but the number of cells per unit
length were scaled proportionally, maintaining the same background mesh cell
aspect ratio. Results for this parameter sweep are shown in 
Figure~\ref{fig:spatial-grid-dep}, from which the final number of streamwise
grid points $N_x = 70$ was chosen.

\begin{figure}[ht]
\caption{Spatial grid dependence for the SST (a) and Spalart--Allmaras (b) 
turbulence models for fixed timesteps.}
\label{fig:spatial-grid-dep}
\end{figure}

Timestep dependence was studied using the $N_x=70$ grid, the results from which
are shown in Figure~\ref{fig:timestep-dep}. It was seen that the
Spalart--Allmaras model converged well with decreasing timestep, leading to a
final timestep of 0.001 s. The results from the SST model show a local minimum,
at $\Delta t = 0.002 s$, which was the final timestep chosen to run the
simulations.

\begin{figure}[ht]
\caption{Spatial grid dependence for the SST (a) and Spalart--Allmaras (b) 
turbulence models for a grid with $N_x=70$.}
\label{fig:timestep-dep}
\end{figure}

\section{Model validation}

Models were validated against several open datasets.


\section{Results}

\subsection{Performance prediction}

Predictions for both the power and drag coefficients are shown in Figure~\ref{fig:perf-comp}

\begin{figure}[ht]
\caption{Power (a) and drag (b) coefficient predictions from experiments and
each numerical model.}
\label{fig:perf-comp}
\end{figure}

\todo[inline]{Add bar graph comparing various performance results for the RVAT.}





\begin{acknowledgments}
The authors acknowledge Vincent Neary and the Sandia National Laboratories Water
Power Program for use of their Red Mesa high performance computing cluster.
\end{acknowledgments}

% Create the reference section using BibTeX:
\bibliography{library}

\end{document}
